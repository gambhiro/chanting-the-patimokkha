\documentclass[10pt,landscape]{article}

\makeatletter

\usepackage{graphicx}
\graphicspath{{./images/}}
\usepackage{xcolor}

\usepackage{fontspec}
\setmainfont[Ligatures=TeX]{TeX Gyre Pagella}

\usepackage{amsmath}
\usepackage{calc}
\usepackage{eso-pic}

\usepackage{tikz}
\usetikzlibrary{positioning}

\usepackage{pgfplots}
\usepgfplotslibrary{dateplot}

\usepackage{pgfcalendar}
\usepackage{pgfplotstable}

\usepackage{geometry}
\geometry{paper=a5paper, landscape}

\pgfplotstableset{string type,col sep=comma}

\pagestyle{empty}

\pgfplotsset{compat=1.9}

\tikzset{
  every pin/.style={draw=black!70,fill=yellow!50!white,rectangle,rounded corners=3pt,font=\small,opacity=0.9},
  small dot/.style={fill=black,circle,scale=0.3}
}

\makeatother

\begin{document}
\mbox{}

\AddToShipoutPictureFG*{\put(\LenToUnit{0.5cm},\LenToUnit{0.5cm}){%
\begin{minipage}[b][\paperheight - 1cm]{\paperwidth - 1cm}
\begin{tikzpicture}
\begin{axis}[
    title=Dive into the Pāṭimokkha,
    width=20cm,
    height=12cm,
    clip=false,
    date coordinates in=x,
    cycle list={%
      {dashed,mark=x},
      {red,mark=*,mark size=1pt},
      {blue},
      {},
    },
    y dir=reverse,
    ylabel=Depth,
    xtick={2014-01-01},
    xticklabel style={rotate=30,anchor=north east,font=\small},
    xticklabel={\month-\day},
    extra x ticks={2014-07-11,2014-10-08,2014-11-21},
    extra x tick labels={Āsāḷhā 07-11,Pavāraṇā 10-08,$s = 0.23$ 11-21},
    every axis legend/.append style={ at={(0.02,0.2)}, anchor=south west },
    legend style={draw=none},
    unbounded coords=jump,
    ytick={0,14,24,32,54,84,132},
    extra y ticks={162},
    extra y tick style={grid=major},
  ]

  \draw[green!40,thick] (axis cs:2014-01-28,24) -- (axis cs:2014-11-21,162);% s = 0.2323 page / day, or 4.3043 day / page

  % ((100-74)/2)/24
  % s = 0.54 page / day, or 1.84 day / page
  \draw[green!40,thick] (axis cs:2014-05-18,74) -- (axis cs:2014-06-11,100);

  % (40/2)/28
  % s = 0.7142, or 1.4
  \draw[green!40,thick] (axis cs:2014-10-20,122) -- (axis cs:2014-11-17,162);

  \addplot table[x=date,y=read] {./plotdata/patimokkha-learning.csv}; \addlegendentry{read};
  \addplot table[x=date,y=recall] {./plotdata/patimokkha-learning.csv}; \addlegendentry{recall};
  \addplot table[x=date,y=recite] {./plotdata/patimokkha-learning.csv}; \addlegendentry{recite};
  \addplot[
    const plot,
    draw=none,
    fill=gray,
    fill opacity=0.2,
    area legend,
  ] table[x=date,y=pain] {./plotdata/patimokkha-learning.csv} \closedcycle; \addlegendentry{pain};

  \draw
    (axis cs:2014-03-01,38.5) -|
    (axis cs:2014-03-20,47.7)
    node[above right] {$\text{sinking} = \frac{\frac{1}{2} dx}{dy} = 0.23 \frac{\text{page}}{\text{day}}$};

  \draw
    (axis cs:2014-06-01,90) |-
    (axis cs:2014-06-10,100)
    node[near start, left] {$0.54 \frac{\text{p}}{\text{d}}$};

  \draw
    (axis cs:2014-10-30,137) |-
    (axis cs:2014-11-08,149)
    node[near start, left] {$0.71 \frac{\text{p}}{\text{d}}$};

  \draw[gray, thin] (axis cs:2014-06-21,78) -- (axis cs:2014-06-21,110);
  \draw[gray, thin] (axis cs:2014-07-11,78) -- (axis cs:2014-07-11,110);
  \draw[<->] (axis cs:2014-06-21,80) -- (axis cs:2014-07-11,80) node[pos=0.5,above] {\small @home};

  \draw[gray, thin] (axis cs:2014-07-21,98) -- (axis cs:2014-07-21,116);
  \draw[gray, thin] (axis cs:2014-07-29,98) -- (axis cs:2014-07-29,116);
  \draw[<->] (axis cs:2014-07-21,100) -- (axis cs:2014-07-29,100) node[pos=0.5,above] {\small @ABM};

  \node (bottomright) [coordinate] at (axis cs:2014-01-01,162) {};

  \node (np10) [coordinate,pin=30:{NP 10: Sea monster!}] at (axis cs:2014-04-22,64) {};
  %\node (np10) [coordinate,pin=30:{NP 10: Level Boss!}] at (axis cs:2014-04-22,64) {};
  %\node[above right=-15pt and 3.15cm of np10.east,opacity=0.7] {\includegraphics[width=2.5cm,angle=-10]{squid-bw.png}};

\end{axis}

%\node (notes) [above left=0pt and 15mm of bottomright, anchor=south west] {%
%  \begin{minipage}{4.5cm}
%  \raggedright\footnotesize
%  324 days. Read is \ldots. Recall is \ldots. Recite is \ldots. The faster the dive, the less the pain (inverse of the derivate of the sum of progress).
%\end{minipage}%
%};

\end{tikzpicture}
\vfill\par\mbox{}
\end{minipage}%
}}

\end{document}
